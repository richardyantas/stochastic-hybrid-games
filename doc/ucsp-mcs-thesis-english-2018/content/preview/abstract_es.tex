\begin{resumen}


% \textsc{traducir despues de definir el english abstract ..........}
Los sistemas híbridos estocásticos son ampliamente usados en la industria
energética, problemas como optimización y seguridad son algunos de los 
mas imporantes. Implementamos una metodologia para mejorar un sistema hibrido
estocastico en tiempo real combinando verificacion de modelos, aprendizaje
por refuerzo y computacion de vector de intervalos. Aplicamos metodos formales
para sintetizar (ej. derivar automaticamente ) un controlador seguro y cercano 
al óptimo. Se modeló un calentador de agua solar híbrido, que incluye 
la interacción humana con el sistema para su consumo de agua caliente, 
como una variable estocástica. Adicionalmente, se empleo \textsc{Uppaal
 Stratego} como una herramienta de optimización que nos permite sintetizar 
 una estrategia de control optimo, minimizando la distancia de separación 
 de la temperatura deseada con la controlada
asi como garantizar limites para su evolución en el tiempo.

\begin{flushleft}
\textbf{Palabras clave:} Sistemas Híbridos Estocásticos, Aprendizaje Automático y Verificación de Modelos.
\end{flushleft}

\end{resumen}
