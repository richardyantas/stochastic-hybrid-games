
\documentclass[12pt]{article}
\usepackage{lingmacros}
\usepackage{tree-dvips}
\usepackage{mathtools}


\begin{document}


\newcommand{\tcont}{\ensuremath{T(t)}}
\newcommand{\vcont}{\ensuremath{V(t)}}
\newcommand{\tenv}{\ensuremath{T_{env}(t)}}
\newcommand{\tin}{\ensuremath{T_{in}(t)}}
\newcommand{\haux}{\ensuremath{P_{aux}}}
\newcommand{\irradiance}{\ensuremath{I_{e}}}

\newcommand{\energycon}{\ensuremath{E_{c}}}


\section*{DynIbex Library}

They look like this:




\subsection{Main Equation:}

The sthocastic hybrid model for the solar water heating has 3 discrete variables 
such as, $\boldsymbol{r_{n}} \in \{ 0,1 \}$, $\boldsymbol{v_{n}} \in \{0,1\}$ and 
$\boldsymbol{p_{n}} \in \{ 1,2,3\}  $ and where $n = t\tau$ and $\tau = 15$ min  is the period.

\begin{equation}
    \begin{aligned}
        \frac{ d }{dt}\tcont = -\frac{2.8811059759131854e^{-06}(\tcont-\tenv)}{\vcont}  
        - \boldsymbol{v_{n}}.\frac{9.34673995175876e^{-05}(\tcont-\tin)}{\vcont} \\
        - sgn(0.1\boldsymbol{p_{n}} - \vcont)\frac{9.34673995175876e^{-05}(\tcont-\tin)}{\vcont} \\
                        + \boldsymbol{r_{n}}.\frac{0.00048018432931886426}{\vcont} +
                        \frac{8.403225763080125e^{-07}\irradiance}{\vcont} \\
    \end{aligned}
\end{equation}

\begin{equation}
    \frac{d}{dt}\vcont = 0.001(0.1\boldsymbol{p_{n}} - \vcont)    
\end{equation}


\begin{equation}
    \frac{d}{dt}\energycon = k\boldsymbol{r_{n}}0.00048018432931886426
\end{equation}

\subsubsection{DynIbex}

$ \tenv \in [40-45] $, $\irradiance \in [0-900] $ and $ \tin \in [30-35]$






\subsection{Equation 1:}


\begin{equation}
    \frac{dx_{1}}{dt} = \frac{x_{1}}{x_{0}},   x_{1}(0) = [1.0,50.0]
\end{equation}


\begin{equation}
    \frac{dx_{0}}{dt} = 1, x_{0}(0) = [1.0,10.1]
\end{equation}

For a T(period) = 10, we got: 

\subsubsection{Simulation with DynIbex}

step = 1e-5:


$x_{1}(t=10)$ = [ENTIRE] , 
$x_{0}(t=10)$ = [ENTIRE] 

\subsubsection{Simulation with Euler Method} 

step = 1e-5s:


$x_{1}(t=10)$ = [1.09901, 100] , 
$x_{0}(t=10)$ = [2, 11.1] 


\end{document}