\section{Data Collection and Annotation}
\label{sec:corpus}
In order to train the \ac{ML} techniques used in this work, we collected images and manually annotated them to build a ground-truth corpus of map visualizations.
In this section, we describe how we collected our image corpus and what aspects of the visualizations we manually annotated.

\figDataset


\subsection{Image Collection}
\label{subsec:imageCollection}
We collected our map images from three well-known geoscience journals in the field of climate change --- Nature, the Journal of Climate, and Geophysical Research Letters. First, we extracted 2,018 figures from 474 documents in PDF format, using the pdffigures tool~\citep{pdffigures2-2016}. Then, we applied chart type classifier by \citeauthor{Poco2017}~\citep{Poco2017} to select only the map visualizations; in total, we collected 1,351 map images.
We then manually applied two constraints: map images must have a color legend, and the map region must have text labels indicating the latitude and longitude values. We leave to future work the problem of identifying a map projection from projected features only (\eg the shape and configuration of land masses).
%For all the manually selected map images, we manually identify their geo projections used. This analysis shows that the most common projections are \emph{Equirectangular} 46\%, \emph{Robinson} 16\%, and \emph{Miller} 13\%. Given this result, we decided to focus our analysis on map images with these three geo projections. Next, we used a uniform random selection of 100 map images for each projection. In this analysis we identify three colorbar types \emph{discrete}, \emph{continuous} and \emph{quantized continuous}\,---\,from now on, we will call it \emph{quantized}. Again, doing analysis we noticed that a few percent of the collected map images contains discrete colorbars, then, we decided to focus our analysis on the first two colorbar types (continuous and quantized). Finally, we discard the discrete colorbars and randomly select more maps, images for continuous and quantized colorbars to complete the 100 images for each geo projection type. \autoref{tab:corpus} shows a summary of our map visualization corpus. \autoref{fig:dataset} shows some image examples from our corpus; we can see maps images with different geo projections and colorbar types,  and these images can show the world map or a geographical region.

In a preliminary analysis, we identified the projections and color legend types used on each map image. This analysis shows that the most common projections are \emph{Equirectangular} 46\%, \emph{Robinson} 16\%, and \emph{Miller} 13\% (see columns in \autoref{fig:dataset}). In addition, we identified three color legend types: \emph{discrete}, \emph{continuous} and \emph{quantized} (see \autoref{sec:color_defs}). We noticed that a few percent of the collected map images contain discrete color legends. Given these results, we decided to focus on these three projections, as well as the \emph{continuous} and \emph{quantized} legends. Finally, we used a uniform random selection of 100 map images for each projection, including both quantized and continuous legends. \autoref{tab:corpus} shows a summary of our map image corpus. \autoref{fig:dataset} shows some examples from our corpus, and we can see map images with different projections and legend types. Some images span the world, and others focus on a specific region.
After selecting the 300 map images, we annotated each of them, following the process below.

\begin{table}[ht]
\centering
\begin{tabular}{lcccc}
\hline
                   & Equirectangular & Miller & Robinson & Total\\
\hline
Continuous         &  29      &  45  & 37     & 111 \\
Quantized          &  71      &  50  & 63     & 189\\
\hline
Total              &  100     &  100  & 100   & 300\\
\hline
\end{tabular}
\caption[Counts of map visualizations per projection and per color legend type.]{Counts of map visualizations per projection and per color legend type, taken from a corpus of map images extracted from climate change publications.}
\label{tab:corpus}
\end{table}


\subsection{Map Annotation}
\label{subsec:mapAnnotation}
For map regions, we annotate three important pieces of information: (i) the map projection type (\ie Equirectangular, Miller or Robinson), (ii) the map location inside the image as represented by a pixel-coordinate bounding box (the orange rectangle in \autoref{fig:ann:map}), and (iii) the location, content, and role of textual elements. The text role type is one of \emph{latitude}, \emph{longitude} or \emph{other}. In \autoref{fig:ann:map}, we can see textual elements indicated with red and blue boxes, representing latitude and longitude, respectively.

\figAnnotationsMap


\subsection{Legend Annotation}
\label{subsec:legAnnotation}
As described previously, our approach supports two types of color legend. In both cases, we annotate the color bar location (green rectangle in \autoref{fig:ann:leg:cont} and \autoref{fig:ann:leg:quant}) and the textual elements including their image location, text and role (\ie \emph{label}, \emph{other}).
For continuous color legends, we also annotate the minimum and maximum pixel coordinates (yellow circles in \autoref{fig:ann:leg:cont}). Note that we can recover the pixel colors by sampling colors along the red line. 
For quantized color legends, we mark a representative pixel inside each bin. \autoref{fig:ann:leg:quant} shows these selected pixels as yellow circles. 

\figAnnotationsLegend

In the next sections, we explain how the annotated corpus is used to train our techniques in order to recover the visual encoding for each map image.
