\chapter{iGeoMap: A System for Interpretation of Map Visualizations}
\label{ch:igeomap}

Our method can be used to support a variety of applications. For instance, data extraction might be used to improve indexing and searching of map visualizations~\citep{Siegel2016, Walter2013}. We might also generate alternative representations of map visualizations for people with vision impairment. 

In this chapter, we present iGeoMap, a web-based system which uses the inferred visual encoding to enable user-interaction from bitmap images of map visualizations. \autoref{sec:igeomapDesc} presents an overview of our web application. Sections \ref{sec:app_autoRecoloring}, \ref{sec:app_overlays}, \ref{sec:app_captionGen}, \ref{sec:app_reprojection} and \ref{sec:app_dataExtract} show the modules of iGeoMap that are available to improve the reading and understanding of a map visualization.

\figIGMview

\section{iGeoMap Overview}
\label{sec:igeomapDesc}

iGeoMap is a web-based system created to enable the user's interaction over bitmap images of map visualizations, allowing the data querying (see \autoref{fig:igeomap:view}) and the redesign of map visualizations. iGeoMap also includes our pipeline (\autoref{ch:proposal}) in its image processing task.

Given a map visualization image as input, we process it using our pipeline, having as output the spatial information, color information, and visual encoding, that are used in the different modules. If the extracted information has any errors, the user can use the \textit{edition mode} of iGeoMap to manually rectify them (see \autoref{fig:igeomap:modes:edit}). In the \textit{interaction mode} the user can interact over the bitmap image using any available interaction, as we see in \autoref{fig:igeomap:modes:interact}.

\figIGMmodes


\subsection*{User Interface}
iGeoMap has a simple interface to be easy to use. Below we detail each part of its main page.

\figIGMgui

\begin{enumerate}[label=\Roman*)]
 \item\textbf{Tool panel.}    As we see in \autoref{fig:igeomap:gui}(I), the tool panel is placed on the left side of the main page of iGeoMap to allow the access to menus and sub-panels that vary depending on the current visualization mode. Within this panel we have the following options:
 \begin{enumerate}[label=\alph*)]
  \item\textbf{Visualization modes:} \autoref{fig:igeomap:gui}(a) shows that user can select \textit{edition} or \textit{interaction} mode. In \textit{edition mode}, the extracted data is visualized to be manually modified (see \autoref{fig:igeomap:modes:edit}). In \textit{interaction mode}, the marks disappear and the menu of interactions is visible (see \autoref{fig:igeomap:modes:interact}).
  \item\textbf{Menu of interactions:} as we see in \autoref{fig:igeomap:gui}b, it is divided in two sub-menus: \textit{map interactions} which contains the available interactions over the map area, and the \textit{legend interactions} that displays the available interactions over the legend.
  \item\textbf{Extra interactions:} iGeoMap has additional options to redesign a map visualization, such as recoloring or reprojection, that are displayed in this sub-panel (see \autoref{fig:igeomap:gui}c).
 \end{enumerate}
    
 \item\textbf{Gallery panel.} As we see in \autoref{fig:igeomap:gui}(II), the gallery is placed on the right side of the main page and shows a set of images that can be used as input to our system.
 
 \item\textbf{Interaction panel.} This panel is placed on the center of the main page (see \autoref{fig:igeomap:gui}(III)). Within this panel is displayed the input map visualization and the user can interact with the bitmap image, selecting data in the map o colors in the legend.
 
 \item\textbf{Information panel.} It is placed on the bottom side of the interaction panel (see \autoref{fig:igeomap:gui}(IV)). This panel appears when additional information is going to be displayed, \eg caption text for the image or statistical charts, depending mainly on the current interaction.
\end{enumerate}

In the following sections, we present the modules of iGeoMap that can be used to interact from a bitmap image of a map visualization.
