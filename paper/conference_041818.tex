\documentclass[conference]{IEEEtran}


\usepackage{macro}

\IEEEoverridecommandlockouts
% The preceding line is only needed to identify funding in the first footnote. If that is unneeded, please comment it out.
\usepackage{cite}
\usepackage{amsmath,amssymb,amsfonts}
\usepackage{algorithm}
\usepackage{graphicx}
\usepackage{textcomp}
\usepackage{xcolor}

\def\BibTeX{{\rm B\kern-.05em{\sc i\kern-.025em b}\kern-.08em
    T\kern-.1667em\lower.7ex\hbox{E}\kern-.125emX}}
\begin{document}

\title{Safe and Near Optimal Controller Synthesis for Stochastic Hybrid Systems\\
{\footnotesize \textsuperscript{*}N....}
\thanks{Identify applicable funding agency here. If none, delete this.}
}


\maketitle

\begin{abstract}
    Stochastic hybrid systems allow to model the interaction between continuos dynamics,
discrete dynamics and probabilistic uncertainty. Because of their versatility,
stochastic hybrid systems have emerged as a powerful framework for capturing 
the intricacies of complex systems. Motivated by this, considerable research effort 
has been devoted to the development of modeling analysis and control 
methods for stochastic hybrid systems. 
% Goals, results.

\end{abstract}

\begin{IEEEkeywords}
SHG, formatting, style, styling, insert
\end{IEEEkeywords}

\section{Introduction}
Hybrid systems are widely used in engineering applications and its 
importance has grown up considerably these last years, because of their 
ease of implementation for controlling cyber-physical systems.
A switched systems is a set of dynamical systems, each with its own 
dynamical behaviour controlled by a parameter mode $u$ whose values 
are in a finite set. However, due to the composition of many switched 
systems together, the global switched systems has a number of modes 
and dynamics which increases exponentially. Switched systems have 
numerous applications in control of mechanical systems, the automotive
 industry, and many other fields. 
\cite{larsen2016online}.


\section{Hybrid Systems}
In this part is presented a method based on correction by design of
 discrete linear switched system in the time. the method consist of 
 given a objective region \emph{R} of state space, the method built 
 a set \emph{S} and a control that guide any element from  \emph{S} 
 a \emph{R}. This method works in an iterative way to back to reach 
 the region \emph{R}. The method  can also be used for synthesize 
 a stability control that is keep inside of R, whole states start in
  \emph{R}.

\subsection{Sthocastic Hybrid Systems Modeling}


\subsection{Sthocastic Hybrid Systems Safety Definitions}

  \textbf{Problem 1}\emph{(Control Synthesis Problem)}. Let us 
consider a sampled Hybrid System. Given three sets R,S and B, 
with ${R \cup B \in S}$  and ${R \cap B = \varnothing }$ find a 
rule ${\sigma(.)}$ such that, for any ${x(0) \in R }$. 

\begin{itemize}
    \item \emph{ ${\tau}$-stability: ${x(t)}$ return in R 
    infinitely often, at some multiples of sampling time ${\tau}$}.
    \item \emph{ safety: ${x(t)}$ always stays in ${S/B}$.}
\end{itemize}




 \textbf{Problem 2} \emph{((R,S) - Stability Problem)}. Given a 
 switched system, a set of recurrence 
 ${\mathbb{R}^n}$ and a safe set \emph{S} ${\subset \mathbb{R}^n}$,find a control rule 
 ${\sigma : \mathbb{R}^+ \rightarrow U}$ such that, for any
  initial condition ${x_0  \in  R_1}$ and any perturbation 
  ${\varpi :\mathbb{R}^+\rightarrow U}$  the
   following holds:
 
 \begin{itemize}
    \item \emph{ Recurrence in \emph{R}:there are a monotonically 
    strictly increasing sequence of (positive) integers
    ${k_t, t \in \mathbb{N}}$ such that for all ${ t \in \mathbb{R}^n,
    \phi(k_l\tau;t_0,x^0,\sigma,w) \in \mathbb{R} }$.}

    \item \emph{ Stability in \emph{S}: for all ${ t \in \mathbb{R}^n,
    \phi(t;t_0,x^0,\sigma,w) \in S}$ .}
\end{itemize}


 
\textbf{Problem 3} \emph{((${R_1,R_2,S}$) - Reachability problem).
Given a switched system of the form shown above, two sets  
${ R_1 \subset \mathbb{R}^n}$  and ${ R_2 \subset \mathbb{R}^n}$ 
and a safety set  ${S \subset  \mathbb{R}^n}$, find a control rule 
${\sigma}$ :
${\mathbb{R}^+\rightarrow U}$ such that, for any initial condition 
${x_0  \in  R_1}$ and any perturbation  ${\varpi : \mathbb{R}^+  
\rightarrow U}$, the following holds:}

\begin{itemize}
  \item  \emph{Reachability from ${R_1}$ to ${R_2}$: there exists 
  an integer ${k \in \mathbb{N} }$ such that we have ${ \phi( k_l\tau
  ;t_0,x^0,\sigma,w) \in R_2 }$.}
  
  \item \emph{ Stability in S: for all ${ t \in \mathbb{R}^+, 
  \phi(t;t_0,x^0,\sigma,w) \in S}$ .}
\end{itemize}


\section{Case Study: Solar Water Heating}


\subsection{Solar Water Heating as Sthocastic Hybrid Game}

The hybrid solar water heating scenario with 12 modes of operations is
defined  like this: $\mathcal{G}_{n,m} = (\mathcal{C,U,X,F},\delta)$, 
where the controller $\mathcal{C}$ has a finite set of controllable modes,
given by resistance state ${r \in \mathbb{B} = \left\lbrace 0,1 \right\rbrace }$ 
and piston movement $p \in \left\lbrace-1,0,1\right\rbrace $. 
The environment $\mathcal{U}$ has a finite set of uncontrollable modes
 $v \in \mathbb{B} $, that means the valve state for opening/closing
water aperture. We assume that $\mathcal{U}$ given $\delta$ can switch
among modes with equal probability at every period $\tau$. The state variables
in $\mathcal{X}_{(t)}$ are given by $\left\lbrace \tcont,\vcont,\energy \right\rbrace $, 
tank temperature, tank volumen and energy consumption respectively. Also 
Disturbance effect is considered in the dynamical system such as environment
temperature, water input temperature and irradiance as a uncontrollable 
continuos environment variables.


\begin{align}
    \frac{d}{dt}\tcont &=   -\frac{1}{\vcont}  \constone(\tcont-\tenv) - 
    \frac{ \textbf{v} }{\vcont}\consttwo (\tcont-\tin) + \notag\\ &\qquad +
    \textbf{r} \frac{ \constthree \auxheat  }{\vcont}  + 
    \frac{ \constfour \irradiance}{\vcont} 
\label{temperature-equation}
\end{align}

\begin{equation}
    \frac{d}{dt}\vcont = -k \textbf{p}; \quad
% \frac{d}{dt}\vcont = k .sgn(100 \textbf{p}  - \vcont); \quad
\label{volume-equation}
\end{equation}

\begin{equation} 
\frac{d}{dt}\energy =  \textbf{r}  \auxheat ;
\label{Energy-equation}
\end{equation}

In equation \ref{temperature-equation}. the paremeters,  $\left\lbrace 
\constone, \consttwo, \constthree, \constfour \right\rbrace $ 
remains constant in time. In equation \ref{volume-equation}, 
\emph{k} is the piston velocity.


\subsection{Solar Water Heating simulation}\label{AA}


....



As mention before, safety behaviour consist in guaranty behaviour
inside of the box as limits of state variables, as a consequence
we define a pattern as a sequence of operations modes.

\subsection{Strategy Controller Synthesis}
\begin{itemize}
\item Use either SI (MKS) or CGS as primary units. (SI units are encouraged.) English units may be used as secondary units (in parentheses). An exception would be the use of English units as identifiers in trade, such as ``3.5-inch disk drive''.
\item Avoid combining SI and CGS units, such as current in amperes and magnetic field in oersteds. This often leads to confusion because equations do not balance dimensionally. If you must use mixed units, clearly state the units for each quantity that you use in an equation.
\item Do not mix complete spellings and abbreviations of units: ``Wb/m\textsuperscript{2}'' or ``webers per square meter'', not ``webers/m\textsuperscript{2}''. Spell out units when they appear in text: ``. . . a few henries'', not ``. . . a few H''.
\item Use a zero before decimal points: ``0.25'', not ``.25''. Use ``cm\textsuperscript{3}'', not ``cc''.)
\end{itemize}

\subsection{Equations}
Number equations consecutively. To make your 
equations more compact, you may use the solidus (~/~), the exp function, or 
appropriate exponents. Italicize Roman symbols for quantities and variables, 
but not Greek symbols. Use a long dash rather than a hyphen for a minus 
sign. Punctuate equations with commas or periods when they are part of a 
sentence, as in:
\begin{equation}
a+b=\gamma\label{eq}
\end{equation}

Be sure that the 
symbols in your equation have been defined before or immediately following 
the equation. Use ``\eqref{eq}'', not ``Eq.~\eqref{eq}'' or ``equation \eqref{eq}'', except at 
the beginning of a sentence: ``Equation \eqref{eq} is . . .''

\subsection{Some Common Mistakes}\label{SCM}
\begin{itemize}
\item The word ``data'' is plural, not singular.
\item The subscript for the permeability of vacuum $\mu_{0}$, and other common scientific constants, is zero with subscript formatting, not a lowercase letter ``o''.
\item In American English, commas, semicolons, periods, question and exclamation marks are located within quotation marks only when a complete thought or name is cited, such as a title or full quotation. When quotation marks are used, instead of a bold or italic typeface, to highlight a word or phrase, punctuation should appear outside of the quotation marks. A parenthetical phrase or statement at the end of a sentence is punctuated outside of the closing parenthesis (like this). (A parenthetical sentence is punctuated within the parentheses.)
\item A graph within a graph is an ``inset'', not an ``insert''. The word alternatively is preferred to the word ``alternately'' (unless you really mean something that alternates).
\item Do not use the word ``essentially'' to mean ``approximately'' or ``effectively''.
\item In your paper title, if the words ``that uses'' can accurately replace the word ``using'', capitalize the ``u''; if not, keep using lower-cased.
\item Be aware of the different meanings of the homophones ``affect'' and ``effect'', ``complement'' and ``compliment'', ``discreet'' and ``discrete'', ``principal'' and ``principle''.
\item Do not confuse ``imply'' and ``infer''.
\item The prefix ``non'' is not a word; it should be joined to the word it modifies, usually without a hyphen.
\item There is no period after the ``et'' in the Latin abbreviation ``et al.''.
\item The abbreviation ``i.e.'' means ``that is'', and the abbreviation ``e.g.'' means ``for example''.
\end{itemize}


\subsection{Authors and Affiliations}
\textbf{The class file is designed for, but not limited to, six authors.} A 
minimum of one author is required for all conference articles. Author names 
should be listed starting from left to right and then moving down to the 
next line. This is the author sequence that will be used in future citations 
and by indexing services. Names should not be listed in columns nor group by 
affiliation. Please keep your affiliations as succinct as possible (for 
example, do not differentiate among departments of the same organization).

\subsection{Identify the Headings}
Headings, or heads, are organizational devices that guide the reader through 
your paper. There are two types: component heads and text heads.

Component heads identify the different components of your paper and are not 
topically subordinate to each other. Examples include Acknowledgments and 
References and, for these, the correct style to use is ``Heading 5''. Use 
``figure caption'' for your Figure captions, and ``table head'' for your 
table title. Run-in heads, such as ``Abstract'', will require you to apply a 
style (in this case, italic) in addition to the style provided by the drop 
down menu to differentiate the head from the text.

Text heads organize the topics on a relational, hierarchical basis. For 
example, the paper title is the primary text head because all subsequent 
material relates and elaborates on this one topic. If there are two or more 
sub-topics, the next level head (uppercase Roman numerals) should be used 
and, conversely, if there are not at least two sub-topics, then no subheads 
should be introduced.

\subsection{Figures and Tables}
\paragraph{Positioning Figures and Tables} Place figures and tables at the top and 
bottom of columns. Avoid placing them in the middle of columns. Large 
figures and tables may span across both columns. Figure captions should be 
below the figures; table heads should appear above the tables. Insert 
figures and tables after they are cited in the text. Use the abbreviation 
``Fig.~\ref{fig}'', even at the beginning of a sentence.

\begin{table}[htbp]
\caption{Table Type Styles}
\begin{center}
\begin{tabular}{|c|c|c|c|}
\hline
\textbf{Table}&\multicolumn{3}{|c|}{\textbf{Table Column Head}} \\
\cline{2-4} 
\textbf{Head} & \textbf{\textit{Table column subhead}}& \textbf{\textit{Subhead}}& \textbf{\textit{Subhead}} \\
\hline
copy& More table copy$^{\mathrm{a}}$& &  \\
\hline
\multicolumn{4}{l}{$^{\mathrm{a}}$Sample of a Table footnote.}
\end{tabular}
\label{tab1}
\end{center}
\end{table}

%\begin{figure}[htbp]
%\centerline{\includegraphics{fig1.png}}
%\caption{Example of a figure caption.}
%\label{fig}
%\end{figure}


Figure Labels: Use 8 point Times New Roman for Figure labels. Use words 
rather than symbols or abbreviations when writing Figure axis labels to 
avoid confusing the reader. As an example, write the quantity 
``Magnetization'', or ``Magnetization, M'', not just ``M''. If including 
units in the label, present them within parentheses. Do not label axes only 
with units. In the example, write ``Magnetization (A/m)'' or ``Magnetization 
\{A[m(1)]\}'', not just ``A/m''. Do not label axes with a ratio of 
quantities and units. For example, write ``Temperature (K)'', not 
``Temperature/K''.

\section{Experiments and Results}

RESULTS ANS EXPERIMENTS SECTIONS, are equal to $\left\lbrace 2.44e^{-5},  4.77e^{-6}, 0.0024, 0.01  \right\rbrace$ respectively. 



The preferred spelling of the word ``acknowledgment'' in America is without 
an ``e'' after the ``g''. Avoid the stilted expression ``one of us (R. B. 
G.) thanks $\ldots$''. Instead, try ``R. B. G. thanks$\ldots$''. Put sponsor 
acknowledgments in the unnumbered footnote on the first page.\cite{larsen2016online}

\section{References}

\bibliographystyle{plain}  %Internal Style
\bibliography{mybib}
\end{document}
%\addcontentsline{toc}{chapter}{Bibliography}
